% Options for packages loaded elsewhere
\PassOptionsToPackage{unicode}{hyperref}
\PassOptionsToPackage{hyphens}{url}
%
\documentclass[
]{article}
\usepackage{lmodern}
\usepackage{amssymb,amsmath}
\usepackage{ifxetex,ifluatex}
\ifnum 0\ifxetex 1\fi\ifluatex 1\fi=0 % if pdftex
  \usepackage[T1]{fontenc}
  \usepackage[utf8]{inputenc}
  \usepackage{textcomp} % provide euro and other symbols
\else % if luatex or xetex
  \usepackage{unicode-math}
  \defaultfontfeatures{Scale=MatchLowercase}
  \defaultfontfeatures[\rmfamily]{Ligatures=TeX,Scale=1}
\fi
% Use upquote if available, for straight quotes in verbatim environments
\IfFileExists{upquote.sty}{\usepackage{upquote}}{}
\IfFileExists{microtype.sty}{% use microtype if available
  \usepackage[]{microtype}
  \UseMicrotypeSet[protrusion]{basicmath} % disable protrusion for tt fonts
}{}
\makeatletter
\@ifundefined{KOMAClassName}{% if non-KOMA class
  \IfFileExists{parskip.sty}{%
    \usepackage{parskip}
  }{% else
    \setlength{\parindent}{0pt}
    \setlength{\parskip}{6pt plus 2pt minus 1pt}}
}{% if KOMA class
  \KOMAoptions{parskip=half}}
\makeatother
\usepackage{xcolor}
\IfFileExists{xurl.sty}{\usepackage{xurl}}{} % add URL line breaks if available
\IfFileExists{bookmark.sty}{\usepackage{bookmark}}{\usepackage{hyperref}}
\hypersetup{
  hidelinks,
  pdfcreator={LaTeX via pandoc}}
\urlstyle{same} % disable monospaced font for URLs
\usepackage[margin=1in]{geometry}
\usepackage{color}
\usepackage{fancyvrb}
\newcommand{\VerbBar}{|}
\newcommand{\VERB}{\Verb[commandchars=\\\{\}]}
\DefineVerbatimEnvironment{Highlighting}{Verbatim}{commandchars=\\\{\}}
% Add ',fontsize=\small' for more characters per line
\usepackage{framed}
\definecolor{shadecolor}{RGB}{248,248,248}
\newenvironment{Shaded}{\begin{snugshade}}{\end{snugshade}}
\newcommand{\AlertTok}[1]{\textcolor[rgb]{0.94,0.16,0.16}{#1}}
\newcommand{\AnnotationTok}[1]{\textcolor[rgb]{0.56,0.35,0.01}{\textbf{\textit{#1}}}}
\newcommand{\AttributeTok}[1]{\textcolor[rgb]{0.77,0.63,0.00}{#1}}
\newcommand{\BaseNTok}[1]{\textcolor[rgb]{0.00,0.00,0.81}{#1}}
\newcommand{\BuiltInTok}[1]{#1}
\newcommand{\CharTok}[1]{\textcolor[rgb]{0.31,0.60,0.02}{#1}}
\newcommand{\CommentTok}[1]{\textcolor[rgb]{0.56,0.35,0.01}{\textit{#1}}}
\newcommand{\CommentVarTok}[1]{\textcolor[rgb]{0.56,0.35,0.01}{\textbf{\textit{#1}}}}
\newcommand{\ConstantTok}[1]{\textcolor[rgb]{0.00,0.00,0.00}{#1}}
\newcommand{\ControlFlowTok}[1]{\textcolor[rgb]{0.13,0.29,0.53}{\textbf{#1}}}
\newcommand{\DataTypeTok}[1]{\textcolor[rgb]{0.13,0.29,0.53}{#1}}
\newcommand{\DecValTok}[1]{\textcolor[rgb]{0.00,0.00,0.81}{#1}}
\newcommand{\DocumentationTok}[1]{\textcolor[rgb]{0.56,0.35,0.01}{\textbf{\textit{#1}}}}
\newcommand{\ErrorTok}[1]{\textcolor[rgb]{0.64,0.00,0.00}{\textbf{#1}}}
\newcommand{\ExtensionTok}[1]{#1}
\newcommand{\FloatTok}[1]{\textcolor[rgb]{0.00,0.00,0.81}{#1}}
\newcommand{\FunctionTok}[1]{\textcolor[rgb]{0.00,0.00,0.00}{#1}}
\newcommand{\ImportTok}[1]{#1}
\newcommand{\InformationTok}[1]{\textcolor[rgb]{0.56,0.35,0.01}{\textbf{\textit{#1}}}}
\newcommand{\KeywordTok}[1]{\textcolor[rgb]{0.13,0.29,0.53}{\textbf{#1}}}
\newcommand{\NormalTok}[1]{#1}
\newcommand{\OperatorTok}[1]{\textcolor[rgb]{0.81,0.36,0.00}{\textbf{#1}}}
\newcommand{\OtherTok}[1]{\textcolor[rgb]{0.56,0.35,0.01}{#1}}
\newcommand{\PreprocessorTok}[1]{\textcolor[rgb]{0.56,0.35,0.01}{\textit{#1}}}
\newcommand{\RegionMarkerTok}[1]{#1}
\newcommand{\SpecialCharTok}[1]{\textcolor[rgb]{0.00,0.00,0.00}{#1}}
\newcommand{\SpecialStringTok}[1]{\textcolor[rgb]{0.31,0.60,0.02}{#1}}
\newcommand{\StringTok}[1]{\textcolor[rgb]{0.31,0.60,0.02}{#1}}
\newcommand{\VariableTok}[1]{\textcolor[rgb]{0.00,0.00,0.00}{#1}}
\newcommand{\VerbatimStringTok}[1]{\textcolor[rgb]{0.31,0.60,0.02}{#1}}
\newcommand{\WarningTok}[1]{\textcolor[rgb]{0.56,0.35,0.01}{\textbf{\textit{#1}}}}
\usepackage{longtable,booktabs}
% Correct order of tables after \paragraph or \subparagraph
\usepackage{etoolbox}
\makeatletter
\patchcmd\longtable{\par}{\if@noskipsec\mbox{}\fi\par}{}{}
\makeatother
% Allow footnotes in longtable head/foot
\IfFileExists{footnotehyper.sty}{\usepackage{footnotehyper}}{\usepackage{footnote}}
\makesavenoteenv{longtable}
\usepackage{graphicx,grffile}
\makeatletter
\def\maxwidth{\ifdim\Gin@nat@width>\linewidth\linewidth\else\Gin@nat@width\fi}
\def\maxheight{\ifdim\Gin@nat@height>\textheight\textheight\else\Gin@nat@height\fi}
\makeatother
% Scale images if necessary, so that they will not overflow the page
% margins by default, and it is still possible to overwrite the defaults
% using explicit options in \includegraphics[width, height, ...]{}
\setkeys{Gin}{width=\maxwidth,height=\maxheight,keepaspectratio}
% Set default figure placement to htbp
\makeatletter
\def\fps@figure{htbp}
\makeatother
\setlength{\emergencystretch}{3em} % prevent overfull lines
\providecommand{\tightlist}{%
  \setlength{\itemsep}{0pt}\setlength{\parskip}{0pt}}
\setcounter{secnumdepth}{-\maxdimen} % remove section numbering

\title{Tipologia i cicle de vida de les dades\\
Pràctica 2. Neteja i anàlisi de dades\\
Solució}
\author{Aitor Ferrus Blasco {[}aferrus{]}\\
Alonso López i Vicente {[}alopezvic{]}}
\date{05/01/2021}

\begin{document}
\maketitle

{
\setcounter{tocdepth}{3}
\tableofcontents
}
\newpage

\hypertarget{descripciuxf3-del-dataset.}{%
\section{1. Descripció del dataset.}\label{descripciuxf3-del-dataset.}}

Perquè és important i quina pregunta/problema pretén respondre?

\textbf{Resposta}

El dataset que hem escollit és \emph{Rain in Australia}
(\url{https://www.kaggle.com/jsphyg/weather-dataset-rattle-package}).

Conté 10 anys de dades d'observacions diàries del clima en diferents
llocs d'Australia. Conté una variable objectiu (RainTomorrow) per predir
el temps del dia següent. Si la variable és Yes indica que ha plogut el
dia següent 1mm o més. Amb aquesta variable podem entrenar models per
tal de predir si plourà el dia següent.

Les variables que inclou el dataset són les següents:

\begin{longtable}[]{@{}ll@{}}
\toprule
\begin{minipage}[b]{0.19\columnwidth}\raggedright
Variable\strut
\end{minipage} & \begin{minipage}[b]{0.75\columnwidth}\raggedright
Descripció\strut
\end{minipage}\tabularnewline
\midrule
\endhead
\begin{minipage}[t]{0.19\columnwidth}\raggedright
Date\strut
\end{minipage} & \begin{minipage}[t]{0.75\columnwidth}\raggedright
La data de l'observació\strut
\end{minipage}\tabularnewline
\begin{minipage}[t]{0.19\columnwidth}\raggedright
Location\strut
\end{minipage} & \begin{minipage}[t]{0.75\columnwidth}\raggedright
El nom de la localització de l'estació metereològica.\strut
\end{minipage}\tabularnewline
\begin{minipage}[t]{0.19\columnwidth}\raggedright
MinTemp\strut
\end{minipage} & \begin{minipage}[t]{0.75\columnwidth}\raggedright
La temperatura mínima en graus Celsius\strut
\end{minipage}\tabularnewline
\begin{minipage}[t]{0.19\columnwidth}\raggedright
MaxTemp\strut
\end{minipage} & \begin{minipage}[t]{0.75\columnwidth}\raggedright
La temperatura màxima en graus Celsius\strut
\end{minipage}\tabularnewline
\begin{minipage}[t]{0.19\columnwidth}\raggedright
Rainfall\strut
\end{minipage} & \begin{minipage}[t]{0.75\columnwidth}\raggedright
La quantitat de pluja registrada durant el dia en mm\strut
\end{minipage}\tabularnewline
\begin{minipage}[t]{0.19\columnwidth}\raggedright
Evaporation\strut
\end{minipage} & \begin{minipage}[t]{0.75\columnwidth}\raggedright
La denominada Class A pan evaporation (mm) durant 24 hores a les
9am\strut
\end{minipage}\tabularnewline
\begin{minipage}[t]{0.19\columnwidth}\raggedright
Sunshine\strut
\end{minipage} & \begin{minipage}[t]{0.75\columnwidth}\raggedright
El nombre d'hores de sol durant el dia.\strut
\end{minipage}\tabularnewline
\begin{minipage}[t]{0.19\columnwidth}\raggedright
WindGustDir\strut
\end{minipage} & \begin{minipage}[t]{0.75\columnwidth}\raggedright
La direcció de la ratxa de vent més forta en les 24 hores fins la
mitjanit\strut
\end{minipage}\tabularnewline
\begin{minipage}[t]{0.19\columnwidth}\raggedright
WindGustSpeed\strut
\end{minipage} & \begin{minipage}[t]{0.75\columnwidth}\raggedright
La velocitat (km/h) de la ratxa de vent més forta en les 24 hores fins a
mitjanit\strut
\end{minipage}\tabularnewline
\begin{minipage}[t]{0.19\columnwidth}\raggedright
WindDir9am\strut
\end{minipage} & \begin{minipage}[t]{0.75\columnwidth}\raggedright
Direcció del vent a les 9am\strut
\end{minipage}\tabularnewline
\begin{minipage}[t]{0.19\columnwidth}\raggedright
WindDir3pm\strut
\end{minipage} & \begin{minipage}[t]{0.75\columnwidth}\raggedright
Direcció del vent a les 3pm\strut
\end{minipage}\tabularnewline
\begin{minipage}[t]{0.19\columnwidth}\raggedright
WindSpeed9am\strut
\end{minipage} & \begin{minipage}[t]{0.75\columnwidth}\raggedright
Mitjana de la Velocitat del vent (km/hr) 10 minuts abans de les
9am\strut
\end{minipage}\tabularnewline
\begin{minipage}[t]{0.19\columnwidth}\raggedright
WindSpeed3pm\strut
\end{minipage} & \begin{minipage}[t]{0.75\columnwidth}\raggedright
Mitjana de la Velocitat del vent (km/hr) 10 minuts abans de les
3pm\strut
\end{minipage}\tabularnewline
\begin{minipage}[t]{0.19\columnwidth}\raggedright
Humidity9am\strut
\end{minipage} & \begin{minipage}[t]{0.75\columnwidth}\raggedright
Humitat (percentatge) a les 9am\strut
\end{minipage}\tabularnewline
\begin{minipage}[t]{0.19\columnwidth}\raggedright
Humidity3pm\strut
\end{minipage} & \begin{minipage}[t]{0.75\columnwidth}\raggedright
Humitat (percentatge) a les 3pm\strut
\end{minipage}\tabularnewline
\begin{minipage}[t]{0.19\columnwidth}\raggedright
Pressure9am\strut
\end{minipage} & \begin{minipage}[t]{0.75\columnwidth}\raggedright
Pressió atmosfèrica (hpa) reduïda al nivell mitjà del mar a les
9am\strut
\end{minipage}\tabularnewline
\begin{minipage}[t]{0.19\columnwidth}\raggedright
Pressure3pm\strut
\end{minipage} & \begin{minipage}[t]{0.75\columnwidth}\raggedright
Pressió atmosfèrica (hpa) reduïda al nivell mitjà del mar a les
3pm\strut
\end{minipage}\tabularnewline
\begin{minipage}[t]{0.19\columnwidth}\raggedright
Cloud9am\strut
\end{minipage} & \begin{minipage}[t]{0.75\columnwidth}\raggedright
Fracció del cel enfosquida pels núvols a les 9am. Es mesura en
``oktas'', els quals són una unitat de vuitens. Registre quants hi
ha\strut
\end{minipage}\tabularnewline
\begin{minipage}[t]{0.19\columnwidth}\raggedright
Cloud3pm\strut
\end{minipage} & \begin{minipage}[t]{0.75\columnwidth}\raggedright
Fracció del cel enfosquida pels núvols a les 3pm. Es mesura en
``oktas'', els quals són una unitat de vuits. Registre quants hi
ha\strut
\end{minipage}\tabularnewline
\begin{minipage}[t]{0.19\columnwidth}\raggedright
Temp9am\strut
\end{minipage} & \begin{minipage}[t]{0.75\columnwidth}\raggedright
Temperatura (graus Celsius) a les 9am\strut
\end{minipage}\tabularnewline
\begin{minipage}[t]{0.19\columnwidth}\raggedright
Temp3pm\strut
\end{minipage} & \begin{minipage}[t]{0.75\columnwidth}\raggedright
Temperatura (graus Celsius) a les 3pm\strut
\end{minipage}\tabularnewline
\begin{minipage}[t]{0.19\columnwidth}\raggedright
RainToday\strut
\end{minipage} & \begin{minipage}[t]{0.75\columnwidth}\raggedright
Booleà: 1 si la precipitació (mm) en les 24 hours anteriors a les 9am és
superior a 1mm, sinó 0\strut
\end{minipage}\tabularnewline
\begin{minipage}[t]{0.19\columnwidth}\raggedright
RainTomorrow\strut
\end{minipage} & \begin{minipage}[t]{0.75\columnwidth}\raggedright
La quantitat de pluja al dia següent en mm. Utilitzada per crear la
variable resposta RainTomorrow. Un tipus de mesura del ``risc''.\strut
\end{minipage}\tabularnewline
\bottomrule
\end{longtable}

\hypertarget{integraciuxf3-i-selecciuxf3-de-les-dades-dinteruxe8s-a-analitzar.}{%
\section{2. Integració i selecció de les dades d'interès a
analitzar.}\label{integraciuxf3-i-selecciuxf3-de-les-dades-dinteruxe8s-a-analitzar.}}

\textbf{Resposta}

Hem seleccionat les dades de Melbourne, ja que tenen poques NA. Creiem
que l'anàlisi que es pot realitzar en aquesta localització és pot
adapatar ràpidament a qualsevol de les altres estacions que inclou el
dataset.

\begin{Shaded}
\begin{Highlighting}[]
\KeywordTok{library}\NormalTok{(readr)}
\NormalTok{weatherAUS <-}\StringTok{ }\KeywordTok{read_csv}\NormalTok{(}\StringTok{"../data/weatherAUS.csv"}\NormalTok{, }
    \DataTypeTok{col_types =} \KeywordTok{cols}\NormalTok{(}\DataTypeTok{Date =} \KeywordTok{col_date}\NormalTok{(}\DataTypeTok{format =} \StringTok{"%Y-%m-%d"}\NormalTok{), }
        \DataTypeTok{Evaporation =} \KeywordTok{col_double}\NormalTok{(), }\DataTypeTok{Sunshine =} \KeywordTok{col_double}\NormalTok{()))}
\NormalTok{weatherMelb <-}\StringTok{ }\NormalTok{weatherAUS[weatherAUS}\OperatorTok{$}\NormalTok{Location }\OperatorTok{==}\StringTok{ "Melbourne"}\NormalTok{,]}
\KeywordTok{summary}\NormalTok{(weatherMelb)}
\end{Highlighting}
\end{Shaded}

\begin{verbatim}
##       Date              Location            MinTemp         MaxTemp     
##  Min.   :2008-07-01   Length:3193        Min.   : 1.40   Min.   : 9.70  
##  1st Qu.:2010-09-07   Class :character   1st Qu.: 8.70   1st Qu.:16.10  
##  Median :2013-01-13   Mode  :character   Median :11.40   Median :19.50  
##  Mean   :2013-01-02                      Mean   :11.78   Mean   :20.77  
##  3rd Qu.:2015-04-19                      3rd Qu.:14.60   3rd Qu.:24.20  
##  Max.   :2017-06-25                      Max.   :28.60   Max.   :46.40  
##                                          NA's   :480     NA's   :481    
##     Rainfall      Evaporation       Sunshine      WindGustDir       
##  Min.   : 0.00   Min.   : 0.00   Min.   : 0.000   Length:3193       
##  1st Qu.: 0.00   1st Qu.: 2.20   1st Qu.: 3.100   Class :character  
##  Median : 0.00   Median : 4.00   Median : 6.500   Mode  :character  
##  Mean   : 1.87   Mean   : 4.65   Mean   : 6.385                     
##  3rd Qu.: 1.20   3rd Qu.: 6.40   3rd Qu.: 9.600                     
##  Max.   :82.20   Max.   :23.80   Max.   :13.900                     
##  NA's   :758     NA's   :3       NA's   :1                          
##  WindGustSpeed     WindDir9am         WindDir3pm         WindSpeed9am  
##  Min.   : 11.00   Length:3193        Length:3193        Min.   : 0.00  
##  1st Qu.: 33.00   Class :character   Class :character   1st Qu.:11.00  
##  Median : 43.00   Mode  :character   Mode  :character   Median :17.00  
##  Mean   : 45.61                                         Mean   :19.13  
##  3rd Qu.: 56.00                                         3rd Qu.:26.00  
##  Max.   :122.00                                         Max.   :67.00  
##  NA's   :14                                             NA's   :2      
##   WindSpeed3pm   Humidity9am      Humidity3pm      Pressure9am    
##  Min.   : 0.0   Min.   : 14.00   Min.   :  6.00   Min.   : 988.9  
##  1st Qu.:15.0   1st Qu.: 58.00   1st Qu.: 41.00   1st Qu.:1012.6  
##  Median :20.0   Median : 68.00   Median : 51.00   Median :1017.9  
##  Mean   :22.1   Mean   : 67.55   Mean   : 51.18   Mean   :1017.6  
##  3rd Qu.:28.0   3rd Qu.: 78.00   3rd Qu.: 61.00   3rd Qu.:1023.0  
##  Max.   :76.0   Max.   :100.00   Max.   :100.00   Max.   :1039.0  
##                 NA's   :482      NA's   :487      NA's   :480     
##   Pressure3pm        Cloud9am        Cloud3pm        Temp9am     
##  Min.   : 988.3   Min.   :0.000   Min.   :0.000   Min.   : 2.90  
##  1st Qu.:1010.7   1st Qu.:3.000   1st Qu.:4.000   1st Qu.:11.28  
##  Median :1016.1   Median :7.000   Median :6.000   Median :14.10  
##  Mean   :1015.8   Mean   :5.314   Mean   :5.336   Mean   :14.60  
##  3rd Qu.:1021.1   3rd Qu.:7.000   3rd Qu.:7.000   3rd Qu.:17.40  
##  Max.   :1035.8   Max.   :8.000   Max.   :8.000   Max.   :35.50  
##  NA's   :483      NA's   :1034    NA's   :1106    NA's   :481    
##     Temp3pm       RainToday         RainTomorrow      
##  Min.   : 7.20   Length:3193        Length:3193       
##  1st Qu.:14.90   Class :character   Class :character  
##  Median :18.20   Mode  :character   Mode  :character  
##  Mean   :19.26                                        
##  3rd Qu.:22.50                                        
##  Max.   :45.40                                        
##  NA's   :484
\end{verbatim}

\hypertarget{neteja-de-les-dades}{%
\section{3. Neteja de les dades}\label{neteja-de-les-dades}}

\hypertarget{les-dades-contenen-zeros-o-elements-buits-com-gestionaries-aquests-casos}{%
\subsection{3.1. Les dades contenen zeros o elements buits? Com
gestionaries aquests
casos?}\label{les-dades-contenen-zeros-o-elements-buits-com-gestionaries-aquests-casos}}

\textbf{Resposta}

\begin{Shaded}
\begin{Highlighting}[]
\CommentTok{# Verifiquem si les dades no tenen valors nulls}
\KeywordTok{sort}\NormalTok{(}\KeywordTok{colMeans}\NormalTok{(}\KeywordTok{is.na}\NormalTok{(weatherMelb)), }\DataTypeTok{decreasing =} \OtherTok{TRUE}\NormalTok{)}
\end{Highlighting}
\end{Shaded}

\begin{verbatim}
##      Cloud3pm      Cloud9am      Rainfall     RainToday  RainTomorrow 
##  0.3463827122  0.3238333855  0.2373943000  0.2373943000  0.2373943000 
##   Humidity3pm       Temp3pm   Pressure3pm   Humidity9am       MaxTemp 
##  0.1525211400  0.1515815847  0.1512683996  0.1509552145  0.1506420294 
##       Temp9am       MinTemp   Pressure9am    WindDir9am   WindGustDir 
##  0.1506420294  0.1503288443  0.1503288443  0.0156592546  0.0043845913 
## WindGustSpeed    WindDir3pm   Evaporation  WindSpeed9am      Sunshine 
##  0.0043845913  0.0037582211  0.0009395553  0.0006263702  0.0003131851 
##          Date      Location  WindSpeed3pm 
##  0.0000000000  0.0000000000  0.0000000000
\end{verbatim}

Les dades contenen elements buits en totes les columnes excepte Date i
Location. Les columnes Cloud3pm , Cloud9am tenen mes de un 30\% de
valors nulls. Aixi que hem decidit que el nombre es molt gran i
exclourem aquestes columnes del nostre dataset.

\begin{Shaded}
\begin{Highlighting}[]
\CommentTok{# Eliminem les  Columnes Cloud3pm i Cloud9am}
\NormalTok{weatherMelb <-}\StringTok{ }\KeywordTok{subset}\NormalTok{( weatherMelb, }\DataTypeTok{select =} \OperatorTok{-}\KeywordTok{c}\NormalTok{(Cloud3pm, Cloud9am ) )}
\end{Highlighting}
\end{Shaded}

\begin{Shaded}
\begin{Highlighting}[]
\CommentTok{# Imputem valors, utilitzem package VIM i funció kNN.}
\KeywordTok{library}\NormalTok{(VIM)}
\end{Highlighting}
\end{Shaded}

\begin{verbatim}
## Loading required package: colorspace
\end{verbatim}

\begin{verbatim}
## Loading required package: grid
\end{verbatim}

\begin{verbatim}
## VIM is ready to use.
\end{verbatim}

\begin{verbatim}
## Suggestions and bug-reports can be submitted at: https://github.com/statistikat/VIM/issues
\end{verbatim}

\begin{verbatim}
## 
## Attaching package: 'VIM'
\end{verbatim}

\begin{verbatim}
## The following object is masked from 'package:datasets':
## 
##     sleep
\end{verbatim}

\begin{Shaded}
\begin{Highlighting}[]
\NormalTok{weatherMelb_complet <-}\StringTok{ }\KeywordTok{kNN}\NormalTok{(weatherMelb)}
\NormalTok{weatherMelb <-}\StringTok{ }\NormalTok{weatherMelb_complet[}\DecValTok{0}\OperatorTok{:}\DecValTok{21}\NormalTok{]}
\end{Highlighting}
\end{Shaded}

Hem utilitzat kNN per a imputar els valors perduts aixi que les nostres
dades no deurien de tenir cap valor null. Ho confirmem:

\begin{Shaded}
\begin{Highlighting}[]
\CommentTok{# Verifiquem que les dades no tenen valors nulls}
\KeywordTok{sort}\NormalTok{(}\KeywordTok{colMeans}\NormalTok{(}\KeywordTok{is.na}\NormalTok{(weatherMelb)), }\DataTypeTok{decreasing =} \OtherTok{TRUE}\NormalTok{)}
\end{Highlighting}
\end{Shaded}

\begin{verbatim}
##          Date      Location       MinTemp       MaxTemp      Rainfall 
##             0             0             0             0             0 
##   Evaporation      Sunshine   WindGustDir WindGustSpeed    WindDir9am 
##             0             0             0             0             0 
##    WindDir3pm  WindSpeed9am  WindSpeed3pm   Humidity9am   Humidity3pm 
##             0             0             0             0             0 
##   Pressure9am   Pressure3pm       Temp9am       Temp3pm     RainToday 
##             0             0             0             0             0 
##  RainTomorrow 
##             0
\end{verbatim}

\hypertarget{identificaciuxf3-i-tractament-de-valors-extrems.}{%
\subsection{3.2. Identificació i tractament de valors
extrems.}\label{identificaciuxf3-i-tractament-de-valors-extrems.}}

\textbf{Resposta}

\begin{Shaded}
\begin{Highlighting}[]
\KeywordTok{par}\NormalTok{(}\DataTypeTok{mfrow=}\KeywordTok{c}\NormalTok{(}\DecValTok{1}\NormalTok{,}\DecValTok{4}\NormalTok{))}
\KeywordTok{boxplot}\NormalTok{(weatherMelb}\OperatorTok{$}\NormalTok{MinTemp, }\DataTypeTok{na.rm=}\OtherTok{TRUE}\NormalTok{, }\DataTypeTok{main=}\StringTok{"Temp mínima"}\NormalTok{)}
\KeywordTok{boxplot}\NormalTok{(weatherMelb}\OperatorTok{$}\NormalTok{MaxTemp, }\DataTypeTok{na.rm=}\OtherTok{TRUE}\NormalTok{, }\DataTypeTok{main=}\StringTok{"Temp màxima"}\NormalTok{)}
\KeywordTok{boxplot}\NormalTok{(weatherMelb}\OperatorTok{$}\NormalTok{Rainfall, }\DataTypeTok{na.rm=}\OtherTok{TRUE}\NormalTok{, }\DataTypeTok{main=}\StringTok{"Pluja"}\NormalTok{)}
\KeywordTok{boxplot}\NormalTok{(weatherMelb}\OperatorTok{$}\NormalTok{Evaporation, }\DataTypeTok{na.rm=}\OtherTok{TRUE}\NormalTok{, }\DataTypeTok{main=}\StringTok{"Evaporació")}
\end{Highlighting}
\end{Shaded}

\includegraphics{PR2_solucio_files/figure-latex/val1-1.pdf}

\begin{Shaded}
\begin{Highlighting}[]
\KeywordTok{par}\NormalTok{(}\DataTypeTok{mfrow=}\KeywordTok{c}\NormalTok{(}\DecValTok{1}\NormalTok{,}\DecValTok{4}\NormalTok{))}
\KeywordTok{boxplot}\NormalTok{(weatherMelb}\OperatorTok{$}\NormalTok{Sunshine, }\DataTypeTok{na.rm=}\OtherTok{TRUE}\NormalTok{, }\DataTypeTok{main=}\StringTok{"Hores de sol"}\NormalTok{)}
\KeywordTok{boxplot}\NormalTok{(weatherMelb}\OperatorTok{$}\NormalTok{WindGustSpeed, }\DataTypeTok{na.rm=}\OtherTok{TRUE}\NormalTok{, }\DataTypeTok{main=}\StringTok{"Ratxa de vent més forta"}\NormalTok{)}
\KeywordTok{boxplot}\NormalTok{(weatherMelb}\OperatorTok{$}\NormalTok{WindSpeed9am, }\DataTypeTok{na.rm=}\OtherTok{TRUE}\NormalTok{, }\DataTypeTok{main=}\StringTok{"Vel. vent 10min abans 9am"}\NormalTok{)}
\KeywordTok{boxplot}\NormalTok{(weatherMelb}\OperatorTok{$}\NormalTok{WindSpeed3pm, }\DataTypeTok{na.rm=}\OtherTok{TRUE}\NormalTok{, }\DataTypeTok{main=}\StringTok{"Vel. vent 10min abans 3pm"}\NormalTok{)}
\end{Highlighting}
\end{Shaded}

\includegraphics{PR2_solucio_files/figure-latex/val2-1.pdf}

\begin{Shaded}
\begin{Highlighting}[]
\KeywordTok{par}\NormalTok{(}\DataTypeTok{mfrow=}\KeywordTok{c}\NormalTok{(}\DecValTok{1}\NormalTok{,}\DecValTok{4}\NormalTok{))}
\KeywordTok{boxplot}\NormalTok{(weatherMelb}\OperatorTok{$}\NormalTok{Humidity9am, }\DataTypeTok{na.rm=}\OtherTok{TRUE}\NormalTok{, }\DataTypeTok{main=}\StringTok{"Humitat % a les 9am"}\NormalTok{)}
\KeywordTok{boxplot}\NormalTok{(weatherMelb}\OperatorTok{$}\NormalTok{Humidity3pm, }\DataTypeTok{na.rm=}\OtherTok{TRUE}\NormalTok{, }\DataTypeTok{main=}\StringTok{"Humitat % a les  3pm"}\NormalTok{)}
\KeywordTok{boxplot}\NormalTok{(weatherMelb}\OperatorTok{$}\NormalTok{Pressure9am, }\DataTypeTok{na.rm=}\OtherTok{TRUE}\NormalTok{, }\DataTypeTok{main=}\StringTok{" Pres. atmos. a les 9am"}\NormalTok{)}
\KeywordTok{boxplot}\NormalTok{(weatherMelb}\OperatorTok{$}\NormalTok{Pressure3pm, }\DataTypeTok{na.rm=}\OtherTok{TRUE}\NormalTok{, }\DataTypeTok{main=}\StringTok{" Pres. atmos. a les 3pm"}\NormalTok{)}
\end{Highlighting}
\end{Shaded}

\includegraphics{PR2_solucio_files/figure-latex/val3-1.pdf}

\begin{Shaded}
\begin{Highlighting}[]
\KeywordTok{par}\NormalTok{(}\DataTypeTok{mfrow=}\KeywordTok{c}\NormalTok{(}\DecValTok{1}\NormalTok{,}\DecValTok{2}\NormalTok{))}
\KeywordTok{boxplot}\NormalTok{(weatherMelb}\OperatorTok{$}\NormalTok{Temp9am, }\DataTypeTok{na.rm=}\OtherTok{TRUE}\NormalTok{, }\DataTypeTok{main=}\StringTok{"Temperatura a les 9am"}\NormalTok{)}
\KeywordTok{boxplot}\NormalTok{(weatherMelb}\OperatorTok{$}\NormalTok{Temp3pm, }\DataTypeTok{na.rm=}\OtherTok{TRUE}\NormalTok{, }\DataTypeTok{main=}\StringTok{"Temperatura a les 3pm"}\NormalTok{)}
\end{Highlighting}
\end{Shaded}

\includegraphics{PR2_solucio_files/figure-latex/val4-1.pdf}

Correcció valors atípics de les columnes MinTemp, MaxTemp , Temp9am i
Temp3pm: Les temperatures màximes d'Austràlia en Melbourne, rarament
passen de 30 graus Celsius i les temperatures mínimes rarament passen de
20 graus Celsius. Per aquesta raó hem decidit corregir els valors
atípics de les columnes MinTemp, MaxTemp , Temp9am i Temp3pm.

També corregim els valors atípics de la columna Evaporation. Al cap de
l'any Australia té una mitja de 1200 mm així que si dividim entre 365
ens ix a 3.2.. Els nombres solen ser majors en estiu i primavera i
menors en la tardor i l'hivern. Així que observant el boxplot les dades
superiors al 12mm semblen ser dades errònies i per tant les hem de
corregir.

Pel que fa a les variables WindGustSpeed, WindSpeed9am i WindSpeed3pm.
És veritat que podem observar certs outliers però, no crec que siguin
dades errònies. Australia és un país que sofreix de tornados cada any
sobretot en les àrees amb gran població com Melbourne així que entenc
que aquestes dades foren extretes durant eixos dies puntuals.

Pel que fa a les variables Humidity9am i Humidity3pm. És veritat que
podem observar certs outliers, però, desprès d'investigar semblem dades
que es poden donar Australia i en cap moment són dades errònies.

Pel que fa a les variables Pressure9am i Pressure3pm. Com anteriorment,
no tinc evidències de què aquest outliers siguin dades errònies per tant
crec que no faria falta tractar-les.

\begin{Shaded}
\begin{Highlighting}[]
\CommentTok{# Apliquem una simple funció per a substituir tots els valors superiors per NA}
\CommentTok{# MinTemp, MaxTemp , Temp9am i Temp3pm.}
\NormalTok{weatherMelb}\OperatorTok{$}\NormalTok{MinTemp <-}\StringTok{ }\KeywordTok{sapply}\NormalTok{(weatherMelb}\OperatorTok{$}\NormalTok{MinTemp, }\ControlFlowTok{function}\NormalTok{(x) }\KeywordTok{ifelse}\NormalTok{(x}\OperatorTok{>}\DecValTok{25}\NormalTok{, }\OtherTok{NA}\NormalTok{, x))}
\NormalTok{weatherMelb}\OperatorTok{$}\NormalTok{MaxTemp <-}\StringTok{ }\KeywordTok{sapply}\NormalTok{(weatherMelb}\OperatorTok{$}\NormalTok{MaxTemp, }\ControlFlowTok{function}\NormalTok{(x) }\KeywordTok{ifelse}\NormalTok{(x}\OperatorTok{>}\DecValTok{35}\NormalTok{, }\OtherTok{NA}\NormalTok{, x))}
\NormalTok{weatherMelb}\OperatorTok{$}\NormalTok{Temp9am <-}\StringTok{ }\KeywordTok{sapply}\NormalTok{(weatherMelb}\OperatorTok{$}\NormalTok{Temp9am, }\ControlFlowTok{function}\NormalTok{(x) }\KeywordTok{ifelse}\NormalTok{(x}\OperatorTok{>}\DecValTok{25}\NormalTok{, }\OtherTok{NA}\NormalTok{, x))}
\NormalTok{weatherMelb}\OperatorTok{$}\NormalTok{Temp3pm <-}\StringTok{ }\KeywordTok{sapply}\NormalTok{(weatherMelb}\OperatorTok{$}\NormalTok{Temp3pm, }\ControlFlowTok{function}\NormalTok{(x) }\KeywordTok{ifelse}\NormalTok{(x}\OperatorTok{>}\DecValTok{32}\NormalTok{, }\OtherTok{NA}\NormalTok{, x))}

\CommentTok{# Evaporation}
\NormalTok{weatherMelb}\OperatorTok{$}\NormalTok{Evaporation <-}\StringTok{ }\KeywordTok{sapply}\NormalTok{(weatherMelb}\OperatorTok{$}\NormalTok{Evaporation, }\ControlFlowTok{function}\NormalTok{(x) }\KeywordTok{ifelse}\NormalTok{(x}\OperatorTok{>}\DecValTok{12}\NormalTok{, }\OtherTok{NA}\NormalTok{, x))}
\end{Highlighting}
\end{Shaded}

\begin{Shaded}
\begin{Highlighting}[]
\CommentTok{# Verifiquem percentaje de valors nulls despres de tractar els outliers}
\KeywordTok{sort}\NormalTok{(}\KeywordTok{colMeans}\NormalTok{(}\KeywordTok{is.na}\NormalTok{(weatherMelb)), }\DataTypeTok{decreasing =} \OtherTok{TRUE}\NormalTok{)}
\end{Highlighting}
\end{Shaded}

\begin{verbatim}
##   Evaporation       Temp3pm       Temp9am       MaxTemp       MinTemp 
##   0.036329471   0.031318509   0.027247103   0.023802067   0.003131851 
##          Date      Location      Rainfall      Sunshine   WindGustDir 
##   0.000000000   0.000000000   0.000000000   0.000000000   0.000000000 
## WindGustSpeed    WindDir9am    WindDir3pm  WindSpeed9am  WindSpeed3pm 
##   0.000000000   0.000000000   0.000000000   0.000000000   0.000000000 
##   Humidity9am   Humidity3pm   Pressure9am   Pressure3pm     RainToday 
##   0.000000000   0.000000000   0.000000000   0.000000000   0.000000000 
##  RainTomorrow 
##   0.000000000
\end{verbatim}

\begin{Shaded}
\begin{Highlighting}[]
\CommentTok{# Imputem valors, utilitzem package VIM i funció kNN.}
\KeywordTok{library}\NormalTok{(VIM)}
\NormalTok{weatherMelb_complet <-}\StringTok{ }\KeywordTok{kNN}\NormalTok{(weatherMelb)}
\NormalTok{weatherMelb <-}\StringTok{ }\NormalTok{weatherMelb_complet[}\DecValTok{0}\OperatorTok{:}\DecValTok{21}\NormalTok{]}
\NormalTok{weatherMelb_complet <-}\StringTok{ }\NormalTok{weatherMelb_complet[}\DecValTok{0}\OperatorTok{:}\DecValTok{21}\NormalTok{]}
\end{Highlighting}
\end{Shaded}

\hypertarget{anuxe0lisi-de-les-dades.}{%
\section{4. Anàlisi de les dades.}\label{anuxe0lisi-de-les-dades.}}

\hypertarget{selecciuxf3-dels-grups-de-dades.}{%
\subsection{4.1. Selecció dels grups de
dades.}\label{selecciuxf3-dels-grups-de-dades.}}

Selecció dels grups de dades que es volen analitzar/comparar
(planificació dels anàlisis a aplicar).

\textbf{Resposta}

Pendent. Cal establir quines dades utilitzarem per predir la pluja.
Cerquem si una (o més) variables ens serveixen per contruir un model que
predigui la pluja al dia següent. L'objectiu és determinar si hi ha una
relació entre les variables.

\begin{Shaded}
\begin{Highlighting}[]
\CommentTok{# Creació variable Month}
\KeywordTok{library}\NormalTok{(dplyr)}
\end{Highlighting}
\end{Shaded}

\begin{verbatim}
## 
## Attaching package: 'dplyr'
\end{verbatim}

\begin{verbatim}
## The following objects are masked from 'package:stats':
## 
##     filter, lag
\end{verbatim}

\begin{verbatim}
## The following objects are masked from 'package:base':
## 
##     intersect, setdiff, setequal, union
\end{verbatim}

\begin{Shaded}
\begin{Highlighting}[]
\NormalTok{weatherMelb <-}\StringTok{ }\NormalTok{weatherMelb }\OperatorTok\StringTok{ }\KeywordTok{mutate}\NormalTok{(}\DataTypeTok{Month =}\NormalTok{ Date)}
\NormalTok{weatherMelb}\OperatorTok{$}\NormalTok{Month<-}\StringTok{ }\KeywordTok{months}\NormalTok{(weatherMelb}\OperatorTok{$}\NormalTok{Month)}



\CommentTok{#fifaNet <- fifaNet %>% mutate(Age_Int = case_when(Age<=20  ~ "Junior",}
\CommentTok{#                                     Age>=21 & Age<=27   ~ "Middle",}
\CommentTok{#                                     Age>=28   ~ "Senior",))}
\end{Highlighting}
\end{Shaded}

\begin{Shaded}
\begin{Highlighting}[]
\CommentTok{#  Creacio variable mesos}
\NormalTok{col <-}\StringTok{ }\KeywordTok{c}\NormalTok{(}\DecValTok{6}\NormalTok{,}\DecValTok{7}\NormalTok{,}\DecValTok{12}\NormalTok{,}\DecValTok{13}\NormalTok{,}\DecValTok{14}\NormalTok{,}\DecValTok{15}\NormalTok{,}\DecValTok{16}\NormalTok{,}\DecValTok{17}\NormalTok{,}\DecValTok{21}\NormalTok{,}\DecValTok{22}\NormalTok{)}
\NormalTok{weatherMelb <-}\StringTok{ }\NormalTok{weatherMelb[col]}
\end{Highlighting}
\end{Shaded}

Hem seleccionat les següents variables: 1. Humidity 2. Presure 3.
WindSpeed 4. Evaporation 5. Sunshine 6. Month

\begin{Shaded}
\begin{Highlighting}[]
\CommentTok{# Podem veure com afecten Humidity + Presure amb la probabilitat de pluja}
\KeywordTok{library}\NormalTok{(ggplot2)}
\KeywordTok{ggplot}\NormalTok{(weatherMelb, }\DataTypeTok{mapping =} \KeywordTok{aes}\NormalTok{(}\DataTypeTok{x =}\NormalTok{ Humidity3pm , }\DataTypeTok{y =}\NormalTok{ Pressure3pm, }\DataTypeTok{color =}\NormalTok{ RainTomorrow) ) }\OperatorTok{+}\StringTok{ }\KeywordTok{geom_line}\NormalTok{()}
\end{Highlighting}
\end{Shaded}

\includegraphics{PR2_solucio_files/figure-latex/Anàlisi de les dades - Humidity + Presure-1.pdf}

\begin{Shaded}
\begin{Highlighting}[]
\KeywordTok{ggplot}\NormalTok{(weatherMelb, }\DataTypeTok{mapping =} \KeywordTok{aes}\NormalTok{(}\DataTypeTok{x =}\NormalTok{ Humidity9am , }\DataTypeTok{y =}\NormalTok{ Pressure9am, }\DataTypeTok{color =}\NormalTok{ RainTomorrow) ) }\OperatorTok{+}\StringTok{ }\KeywordTok{geom_line}\NormalTok{()}
\end{Highlighting}
\end{Shaded}

\includegraphics{PR2_solucio_files/figure-latex/Anàlisi de les dades - Humidity + Presure-2.pdf}
Si la pressió es baixa i la humitat és alta, és un fet clar que una
massa d'aire humida s'acosta i pot estar associada a un front de pluges.
Observant els gràfics podem assumir que el que hem dit abans és
correcte.

\begin{Shaded}
\begin{Highlighting}[]
\CommentTok{# Podem veure com afecten Humidity + WindSpeed amb la probabilitat de pluja}
\KeywordTok{ggplot}\NormalTok{(weatherMelb, }\DataTypeTok{mapping =} \KeywordTok{aes}\NormalTok{(}\DataTypeTok{x =}\NormalTok{ Humidity3pm , }\DataTypeTok{y =}\NormalTok{ WindSpeed3pm, }\DataTypeTok{color =}\NormalTok{ RainTomorrow) ) }\OperatorTok{+}\StringTok{ }\KeywordTok{geom_line}\NormalTok{()}
\end{Highlighting}
\end{Shaded}

\includegraphics{PR2_solucio_files/figure-latex/Anàlisi de les dades 2 - Humidity + WindSpeed-1.pdf}

\begin{Shaded}
\begin{Highlighting}[]
\KeywordTok{ggplot}\NormalTok{(weatherMelb, }\DataTypeTok{mapping =} \KeywordTok{aes}\NormalTok{(}\DataTypeTok{x =}\NormalTok{ Humidity9am , }\DataTypeTok{y =}\NormalTok{ WindSpeed9am, }\DataTypeTok{color =}\NormalTok{ RainTomorrow) ) }\OperatorTok{+}\StringTok{ }\KeywordTok{geom_line}\NormalTok{()}
\end{Highlighting}
\end{Shaded}

\includegraphics{PR2_solucio_files/figure-latex/Anàlisi de les dades 2 - Humidity + WindSpeed-2.pdf}
Un altre factor important és el vent, ja que l'aire fred no saturat
absorbeix la humitat amb molta eficàcia. Als gràfics mostrats
prèviament, podem veure una tendència en la qual velocitats de vent
petites amb mesures d'humitats petites donen lloc a la no pluja a
l'endemà mentre que com més incrementem aquestes dues variables la
probabilitat de pluja sembla augmentar considerablement.

\begin{Shaded}
\begin{Highlighting}[]
\CommentTok{# Podem veure com afecten Humidity + Evaporation amb la probabilitat de pluja}
\KeywordTok{ggplot}\NormalTok{(weatherMelb, }\DataTypeTok{mapping =} \KeywordTok{aes}\NormalTok{(}\DataTypeTok{x =}\NormalTok{ Humidity3pm , }\DataTypeTok{y =}\NormalTok{ Evaporation, }\DataTypeTok{color =}\NormalTok{ RainTomorrow) ) }\OperatorTok{+}\StringTok{ }\KeywordTok{geom_line}\NormalTok{()}
\end{Highlighting}
\end{Shaded}

\includegraphics{PR2_solucio_files/figure-latex/Anàlisi de les dades 3 - Humidity + Evaporation-1.pdf}

\begin{Shaded}
\begin{Highlighting}[]
\KeywordTok{ggplot}\NormalTok{(weatherMelb, }\DataTypeTok{mapping =} \KeywordTok{aes}\NormalTok{(}\DataTypeTok{x =}\NormalTok{ Humidity9am , }\DataTypeTok{y =}\NormalTok{ Evaporation, }\DataTypeTok{color =}\NormalTok{ RainTomorrow) ) }\OperatorTok{+}\StringTok{ }\KeywordTok{geom_line}\NormalTok{()}
\end{Highlighting}
\end{Shaded}

\includegraphics{PR2_solucio_files/figure-latex/Anàlisi de les dades 3 - Humidity + Evaporation-2.pdf}
Quan la massa d'aire no està saturada (humitat relativa del 100\%) la
quantitat d'aigua evaporada es compensa amb una quantitat d'aigua igual
condensada. En les nostres dades desafortunadament no tenim la humitat
relativa i per tant no podem fer una comparació correcta. A més si
observem la gràfica no hi ha cap relació entre la Evaporacio i humitat
que provoqui l'augmente de les probabilitats de pluja a l'endemà tan
sols podem observar que a més humitat registrada les probabilitats
augmenten.

\begin{Shaded}
\begin{Highlighting}[]
\CommentTok{# Podem veure com afecten Humidity + Sunshine amb la probabilitat de pluja}
\KeywordTok{ggplot}\NormalTok{(weatherMelb, }\DataTypeTok{mapping =} \KeywordTok{aes}\NormalTok{(}\DataTypeTok{x =}\NormalTok{ Humidity3pm , }\DataTypeTok{y =}\NormalTok{ Sunshine, }\DataTypeTok{color =}\NormalTok{ RainTomorrow) ) }\OperatorTok{+}\StringTok{ }\KeywordTok{geom_line}\NormalTok{()}
\end{Highlighting}
\end{Shaded}

\includegraphics{PR2_solucio_files/figure-latex/Anàlisi de les dades 4 - Humidity + Sunshine-1.pdf}

\begin{Shaded}
\begin{Highlighting}[]
\KeywordTok{ggplot}\NormalTok{(weatherMelb, }\DataTypeTok{mapping =} \KeywordTok{aes}\NormalTok{(}\DataTypeTok{x =}\NormalTok{ Humidity9am , }\DataTypeTok{y =}\NormalTok{ Sunshine, }\DataTypeTok{color =}\NormalTok{ RainTomorrow) ) }\OperatorTok{+}\StringTok{ }\KeywordTok{geom_line}\NormalTok{()}
\end{Highlighting}
\end{Shaded}

\includegraphics{PR2_solucio_files/figure-latex/Anàlisi de les dades 4 - Humidity + Sunshine-2.pdf}
En aquest grafic podem observar que a mennys hores de sol i mes humitat,
les probabiliyats de plutja al sandema augmenten, mentre que a mes hores
de sol i menys humitat es probabiliyats de plutja al sandema
disminuixen.

\begin{Shaded}
\begin{Highlighting}[]
\NormalTok{p <-}\StringTok{ }\KeywordTok{ggplot}\NormalTok{(}\DataTypeTok{data=}\NormalTok{weatherMelb,}\KeywordTok{aes}\NormalTok{(}\DataTypeTok{x=}\NormalTok{Month,}\DataTypeTok{fill=}\NormalTok{RainTomorrow))}\OperatorTok{+}\KeywordTok{geom_bar}\NormalTok{()}
\NormalTok{p }\OperatorTok{+}\StringTok{ }\KeywordTok{theme}\NormalTok{(}\DataTypeTok{axis.text.x =} \KeywordTok{element_text}\NormalTok{(}\DataTypeTok{angle =} \DecValTok{45}\NormalTok{, }\DataTypeTok{hjust =} \DecValTok{1}\NormalTok{))}
\end{Highlighting}
\end{Shaded}

\includegraphics{PR2_solucio_files/figure-latex/Anàlisi de les dades 4 - Month-1.pdf}

Amb el gràfic anterior podem concloure que en Melbourne no hi ha uns
mesos específics de pluja, sembla que les precipitacions són similars al
llarg de l'any durant els diferents mesos.

\hypertarget{comprovaciuxf3-de-la-normalitat-i-homogeneuxeftat-de-la-variuxe0ncia.}{%
\subsection{4.2. Comprovació de la normalitat i homogeneïtat de la
variància.}\label{comprovaciuxf3-de-la-normalitat-i-homogeneuxeftat-de-la-variuxe0ncia.}}

\textbf{Resposta}

Utilitzem el test de Shapiro-Wilk per comprovar la normalitat. Si el
pvalor és inferior a 0.05, el nivell de significació, podrem rebutjar la
hipòtesi nul·la i concloure que les dades no tenen una distribució
normal. En cas contrari, si el pvalor és major que 0.05 podrem concloure
que les dades segueixen una distribució normal.

\begin{Shaded}
\begin{Highlighting}[]
\CommentTok{# Utilitzem el test de Shapiro-Wilk per comprovar la normalitat de totes les variables cuantitatives}
\KeywordTok{shapiro.test}\NormalTok{(weatherMelb_complet}\OperatorTok{$}\NormalTok{MinTemp)}
\end{Highlighting}
\end{Shaded}

\begin{verbatim}
## 
##  Shapiro-Wilk normality test
## 
## data:  weatherMelb_complet$MinTemp
## W = 0.99356, p-value = 1.061e-10
\end{verbatim}

\begin{Shaded}
\begin{Highlighting}[]
\KeywordTok{shapiro.test}\NormalTok{(weatherMelb_complet}\OperatorTok{$}\NormalTok{MaxTemp)}
\end{Highlighting}
\end{Shaded}

\begin{verbatim}
## 
##  Shapiro-Wilk normality test
## 
## data:  weatherMelb_complet$MaxTemp
## W = 0.95622, p-value < 2.2e-16
\end{verbatim}

\begin{Shaded}
\begin{Highlighting}[]
\KeywordTok{shapiro.test}\NormalTok{(weatherMelb_complet}\OperatorTok{$}\NormalTok{Rainfall)}
\end{Highlighting}
\end{Shaded}

\begin{verbatim}
## 
##  Shapiro-Wilk normality test
## 
## data:  weatherMelb_complet$Rainfall
## W = 0.36727, p-value < 2.2e-16
\end{verbatim}

\begin{Shaded}
\begin{Highlighting}[]
\KeywordTok{shapiro.test}\NormalTok{(weatherMelb_complet}\OperatorTok{$}\NormalTok{Evaporation)}
\end{Highlighting}
\end{Shaded}

\begin{verbatim}
## 
##  Shapiro-Wilk normality test
## 
## data:  weatherMelb_complet$Evaporation
## W = 0.95253, p-value < 2.2e-16
\end{verbatim}

\begin{Shaded}
\begin{Highlighting}[]
\KeywordTok{shapiro.test}\NormalTok{(weatherMelb_complet}\OperatorTok{$}\NormalTok{Sunshine)}
\end{Highlighting}
\end{Shaded}

\begin{verbatim}
## 
##  Shapiro-Wilk normality test
## 
## data:  weatherMelb_complet$Sunshine
## W = 0.95924, p-value < 2.2e-16
\end{verbatim}

\begin{Shaded}
\begin{Highlighting}[]
\KeywordTok{shapiro.test}\NormalTok{(weatherMelb_complet}\OperatorTok{$}\NormalTok{WindGustSpeed)}
\end{Highlighting}
\end{Shaded}

\begin{verbatim}
## 
##  Shapiro-Wilk normality test
## 
## data:  weatherMelb_complet$WindGustSpeed
## W = 0.9667, p-value < 2.2e-16
\end{verbatim}

\begin{Shaded}
\begin{Highlighting}[]
\KeywordTok{shapiro.test}\NormalTok{(weatherMelb_complet}\OperatorTok{$}\NormalTok{WindSpeed9am)}
\end{Highlighting}
\end{Shaded}

\begin{verbatim}
## 
##  Shapiro-Wilk normality test
## 
## data:  weatherMelb_complet$WindSpeed9am
## W = 0.93527, p-value < 2.2e-16
\end{verbatim}

\begin{Shaded}
\begin{Highlighting}[]
\KeywordTok{shapiro.test}\NormalTok{(weatherMelb_complet}\OperatorTok{$}\NormalTok{WindSpeed3pm)}
\end{Highlighting}
\end{Shaded}

\begin{verbatim}
## 
##  Shapiro-Wilk normality test
## 
## data:  weatherMelb_complet$WindSpeed3pm
## W = 0.97805, p-value < 2.2e-16
\end{verbatim}

\begin{Shaded}
\begin{Highlighting}[]
\KeywordTok{shapiro.test}\NormalTok{(weatherMelb_complet}\OperatorTok{$}\NormalTok{Humidity9am)}
\end{Highlighting}
\end{Shaded}

\begin{verbatim}
## 
##  Shapiro-Wilk normality test
## 
## data:  weatherMelb_complet$Humidity9am
## W = 0.98611, p-value < 2.2e-16
\end{verbatim}

\begin{Shaded}
\begin{Highlighting}[]
\KeywordTok{shapiro.test}\NormalTok{(weatherMelb_complet}\OperatorTok{$}\NormalTok{Humidity3pm)}
\end{Highlighting}
\end{Shaded}

\begin{verbatim}
## 
##  Shapiro-Wilk normality test
## 
## data:  weatherMelb_complet$Humidity3pm
## W = 0.99309, p-value = 3.151e-11
\end{verbatim}

\begin{Shaded}
\begin{Highlighting}[]
\KeywordTok{shapiro.test}\NormalTok{(weatherMelb_complet}\OperatorTok{$}\NormalTok{Pressure9am)}
\end{Highlighting}
\end{Shaded}

\begin{verbatim}
## 
##  Shapiro-Wilk normality test
## 
## data:  weatherMelb_complet$Pressure9am
## W = 0.9946, p-value = 1.843e-09
\end{verbatim}

\begin{Shaded}
\begin{Highlighting}[]
\KeywordTok{shapiro.test}\NormalTok{(weatherMelb_complet}\OperatorTok{$}\NormalTok{Pressure3pm)}
\end{Highlighting}
\end{Shaded}

\begin{verbatim}
## 
##  Shapiro-Wilk normality test
## 
## data:  weatherMelb_complet$Pressure3pm
## W = 0.99645, p-value = 7.431e-07
\end{verbatim}

\begin{Shaded}
\begin{Highlighting}[]
\KeywordTok{shapiro.test}\NormalTok{(weatherMelb_complet}\OperatorTok{$}\NormalTok{Temp9am)}
\end{Highlighting}
\end{Shaded}

\begin{verbatim}
## 
##  Shapiro-Wilk normality test
## 
## data:  weatherMelb_complet$Temp9am
## W = 0.99434, p-value = 8.793e-10
\end{verbatim}

\begin{Shaded}
\begin{Highlighting}[]
\KeywordTok{shapiro.test}\NormalTok{(weatherMelb_complet}\OperatorTok{$}\NormalTok{Temp3pm)}
\end{Highlighting}
\end{Shaded}

\begin{verbatim}
## 
##  Shapiro-Wilk normality test
## 
## data:  weatherMelb_complet$Temp3pm
## W = 0.96944, p-value < 2.2e-16
\end{verbatim}

Cap de les variables segueix una distribució normal. El pvalor calculat
és inferior a 0.05, el nivell de significació, així que podem rebutjar
la hipòtesi nul·la i concloure que les dades no tenen una distribució
normal.

Per comprovar la homoscedasticitat, és a dir, la igualtat de variàncies,
podem utilitzar el test de Levene si les dades segueixen una distribució
normal, o el de Fligner-Killen si les dades no segueixen una distribució
normal.

\begin{Shaded}
\begin{Highlighting}[]
\CommentTok{# Utilitzem Fligner-Killen perquè les dades no són normals.}
\KeywordTok{fligner.test}\NormalTok{(Rainfall }\OperatorTok{~}\StringTok{ }\NormalTok{WindSpeed9am, }\DataTypeTok{data =}\NormalTok{ weatherMelb_complet)}
\end{Highlighting}
\end{Shaded}

\begin{verbatim}
## 
##  Fligner-Killeen test of homogeneity of variances
## 
## data:  Rainfall by WindSpeed9am
## Fligner-Killeen:med chi-squared = 106.03, df = 36, p-value = 7.973e-09
\end{verbatim}

\hypertarget{aplicaciuxf3-de-proves-estaduxedstiques.}{%
\subsection{4.3. Aplicació de proves
estadístiques.}\label{aplicaciuxf3-de-proves-estaduxedstiques.}}

Per comparar els grups de dades. En funció de les dades i de l'objectiu
de l'estudi, aplicar proves de contrast d'hipòtesis, correlacions,
regressions, etc. Aplicar almenys tres mètodes d'anàlisi diferents.

\textbf{Resposta}

\hypertarget{representaciuxf3-dels-resultats.}{%
\section{5. Representació dels
resultats.}\label{representaciuxf3-dels-resultats.}}

A partir de taules i gràfiques.

\textbf{Resposta}

\hypertarget{resoluciuxf3-del-problema.}{%
\section{6. Resolució del problema.}\label{resoluciuxf3-del-problema.}}

A partir dels resultats obtinguts, quines són les conclusions? Els
resultats permeten respondre al problema?

\textbf{Resposta}

\hypertarget{codi.}{%
\section{7. Codi.}\label{codi.}}

Cal adjuntar el codi, preferiblement en R, amb el que s'ha realitzat la
neteja, anàlisi i representació de les dades. Si ho preferiu, també
podeu treballar en Python.

\textbf{Resposta}

\hypertarget{contribucions}{%
\section{8. Contribucions}\label{contribucions}}

\begin{longtable}[]{@{}ll@{}}
\toprule
Contribucions & Firma\tabularnewline
\midrule
\endhead
Investigació prèvia & Aitor Ferrus Blasco, Alonso López i
Vicente\tabularnewline
Redacció de les respostes & Aitor Ferrus Blasco, Alonso López i
Vicente\tabularnewline
Desenvolupament codi & Aitor Ferrus Blasco, Alonso López i
Vicente\tabularnewline
\bottomrule
\end{longtable}

\end{document}
